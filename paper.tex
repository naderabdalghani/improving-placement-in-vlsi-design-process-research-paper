\documentclass[conference]{IEEEtran}
\IEEEoverridecommandlockouts
\usepackage{tikz}
\usetikzlibrary{shapes.geometric, arrows}
\usepackage{cite}
\usepackage{amsmath,amssymb,amsfonts}
\usepackage{algorithm}
\usepackage{algpseudocode}
\usepackage{graphicx}
\usepackage[OT1]{fontenc}
\usepackage{xcolor}
\usepackage{pgfplots}
\pgfplotsset{compat=1.9}

\def\BibTeX{{\rm B\kern-.05em{\sc i\kern-.025em b}\kern-.08em
    T\kern-.1667em\lower.7ex\hbox{E}\kern-.125emX}}

\makeatletter

\begin{document}

%%%%%%%%%%%%%%%%%%%%%%%%%%%%%%%%%%%%%%%%%%%%%%%%%%%%%%%%%%%%%%%%%%%%%%%%%%%%%%%%%%%
%%%%%%%%%%%%%%%%%%%%%%%%%%%%%%%%%%%%%% TITLE %%%%%%%%%%%%%%%%%%%%%%%%%%%%%%%%%%%%%%
%%%%%%%%%%%%%%%%%%%%%%%%%%%%%%%%%%%%%%%%%%%%%%%%%%%%%%%%%%%%%%%%%%%%%%%%%%%%%%%%%%%

\title{Paper Title\\}

%%%%%%%%%%%%%%%%%%%%%%%%%%%%%%%%%%%%%%%%%%%%%%%%%%%%%%%%%%%%%%%%%%%%%%%%%%%%%%%%%%%
%%%%%%%%%%%%%%%%%%%%%%%%%%%%%%%%%%%%% AUTHORS %%%%%%%%%%%%%%%%%%%%%%%%%%%%%%%%%%%%%
%%%%%%%%%%%%%%%%%%%%%%%%%%%%%%%%%%%%%%%%%%%%%%%%%%%%%%%%%%%%%%%%%%%%%%%%%%%%%%%%%%%

\author{\IEEEauthorblockN{Nada Adel\IEEEauthorrefmark{1}, Nader AbdAlGhani\IEEEauthorrefmark{2}, Nourhan Gamal\IEEEauthorrefmark{3} and Sarah Raafat\IEEEauthorrefmark{4}}
\IEEEauthorblockA{Computer Engineering Dept.,
Faculty of Engineering, Cairo University\\
Cairo, Egypt\\
\IEEEauthorrefmark{1}nadoza.98@gmail.com,
\IEEEauthorrefmark{2}nader\_abdelghani@hotmail.com,
\IEEEauthorrefmark{3}nourgamal1498@gmail.com,
\IEEEauthorrefmark{4}sararaafat1@yahoo.com}}


\maketitle

%%%%%%%%%%%%%%%%%%%%%%%%%%%%%%%%%%%%%%%%%%%%%%%%%%%%%%%%%%%%%%%%%%%%%%%%%%%%%%%%%%%
%%%%%%%%%%%%%%%%%%%%%%%%%%%%%%%%%%%% ABSTRACT %%%%%%%%%%%%%%%%%%%%%%%%%%%%%%%%%%%%%
%%%%%%%%%%%%%%%%%%%%%%%%%%%%%%%%%%%%%%%%%%%%%%%%%%%%%%%%%%%%%%%%%%%%%%%%%%%%%%%%%%%

\begin{abstract}

Insert abstract text here

\end{abstract}

%%%%%%%%%%%%%%%%%%%%%%%%%%%%%%%%%%%%%%%%%%%%%%%%%%%%%%%%%%%%%%%%%%%%%%%%%%%%%%%%%%%
%%%%%%%%%%%%%%%%%%%%%%%%%%%%%%%%%%%% KEY WORDS %%%%%%%%%%%%%%%%%%%%%%%%%%%%%%%%%%%%
%%%%%%%%%%%%%%%%%%%%%%%%%%%%%%%%%%%%%%%%%%%%%%%%%%%%%%%%%%%%%%%%%%%%%%%%%%%%%%%%%%%

\begin{IEEEkeywords}

Insert keywords here

\end{IEEEkeywords}

%%%%%%%%%%%%%%%%%%%%%%%%%%%%%%%%%%%%%%%%%%%%%%%%%%%%%%%%%%%%%%%%%%%%%%%%%%%%%%%%%%%
%%%%%%%%%%%%%%%%%%%%%%%%%%%%%%%%%% INTRODUCTION %%%%%%%%%%%%%%%%%%%%%%%%%%%%%%%%%%%
%%%%%%%%%%%%%%%%%%%%%%%%%%%%%%%%%%%%%%%%%%%%%%%%%%%%%%%%%%%%%%%%%%%%%%%%%%%%%%%%%%%

\section{Introduction}

The result of the astronomical impact of the very large scale integration (VLSI) industry on the global economy and the extreme widespread use of electronic devices and equipment in various fields across the globe have led to the exponential increase in demand and sophistication of electronic devices. This, of course, goes along with Moore's law which refers to Gordon Moore's projection in 1965 that the number of transistors crammed in integrated circuits (ICs) doubles every two years, though their cost is halved. This increase in the number of transistors put in an IC comes at a cost which are the problems that arise during both the design process and the fabrication process of such intricate devices. This paper addresses the role of placement in the physical design process. Placement tools are used mainly for two things, the first is to place cells in a suitable way for the consequent routing process, and the second is to guide both synthesis and floorplanning processes. Over the years, placement has no longer become a fine-tunning tool used in physical design, but for that matter, it has become a notable contributor to the timing closure process results as well as being an overall major step in physical design. Conventionally, placement is divided into two subprocesses, global placement and detailed placement, however, the scope of this paper will focus exclusively on the former. Global placement aims to evenly and legally distribute the cells over the available placement region putting in mind cells relative positions to each other globally regardless of local problems that may arise, hence the name. This goal is achieved in favour of certain cost metrics such as overall wirelength and the time taken to achieve such a goal. The placement problem is to place the cells in such a way to achieve the best possible cost metrics. Much like the well-known travelling salesman problem and most, if not all, physical design problems, it is an NP-complete problem in combinatorial optimization. Most of the attempts at solving this problem use heuristic algorithms instead of the ineffective brute force trivial approach. Among those metaheuristics are simulated annealing which is popularly used in the optimization of this particular problem, and genetic algorithm which is used for optimization problems in general. In this paper, we implemented the best of both worlds to achieve overall better optimization in terms of wirelength and time.

%%%%%%%%%%%%%%%%%%%%%%%%%%%%%%%%%%%%%%%%%%%%%%%%%%%%%%%%%%%%%%%%%%%%%%%%%%%%%%%%%%%
%%%%%%%%%%%%%%%%%%%%%%%%%%%%%%%%%% RELATED WORK %%%%%%%%%%%%%%%%%%%%%%%%%%%%%%%%%%%
%%%%%%%%%%%%%%%%%%%%%%%%%%%%%%%%%%%%%%%%%%%%%%%%%%%%%%%%%%%%%%%%%%%%%%%%%%%%%%%%%%%

\section{Related Work}

Insert related work here

%%%%%%%%%%%%%%%%%%%%%%%%%%%%%%%%%%%%%%%%%%%%%%%%%%%%%%%%%%%%%%%%%%%%%%%%%%%%%%%%%%%
%%%%%%%%%%%%%%%%%%%%%%%%%%%%%%%% PROPOSED APPROACH %%%%%%%%%%%%%%%%%%%%%%%%%%%%%%%%
%%%%%%%%%%%%%%%%%%%%%%%%%%%%%%%%%%%%%%%%%%%%%%%%%%%%%%%%%%%%%%%%%%%%%%%%%%%%%%%%%%%

\section{Proposed Approach}

Insert proposed approach here

%%%%%%%%%%%%%%%%%%%%%%%%%%%%%%%%%%%%%%%%%%%%%%%%%%%%%%%%%%%%%%%%%%%%%%%%%%%%%%%%%%%
%%%%%%%%%%%%%%%%%%%%%%%%%%%%%% EXPERIMENTAL ANALYSIS %%%%%%%%%%%%%%%%%%%%%%%%%%%%%%
%%%%%%%%%%%%%%%%%%%%%%%%%%%%%%%%%%%%%%%%%%%%%%%%%%%%%%%%%%%%%%%%%%%%%%%%%%%%%%%%%%%

\section{Experimental Analysis}

%%%%%%%%%%%%%%%%%%%%%%%%%%%%%%%%%%%%%%%%%%%%%%%%%%%%%%%%%%%%%%%%%%%%%%%%%%%%%%%%%%%
%%%%%%%%%%%%%%%%%%%%%%%%%%%%%%%%%%% ASSUMPTIONS %%%%%%%%%%%%%%%%%%%%%%%%%%%%%%%%%%%
%%%%%%%%%%%%%%%%%%%%%%%%%%%%%%%%%%%%%%%%%%%%%%%%%%%%%%%%%%%%%%%%%%%%%%%%%%%%%%%%%%%

\subsection{Assumptions}

Insert assumptions here

%%%%%%%%%%%%%%%%%%%%%%%%%%%%%%%%%%%%%%%%%%%%%%%%%%%%%%%%%%%%%%%%%%%%%%%%%%%%%%%%%%%
%%%%%%%%%%%%%%%%%%%%%%%%%%%%%%%%%%%%%%% I/O %%%%%%%%%%%%%%%%%%%%%%%%%%%%%%%%%%%%%%%
%%%%%%%%%%%%%%%%%%%%%%%%%%%%%%%%%%%%%%%%%%%%%%%%%%%%%%%%%%%%%%%%%%%%%%%%%%%%%%%%%%%

\subsection{Experimental Scheme}

Insert I/O here

%%%%%%%%%%%%%%%%%%%%%%%%%%%%%%%%%%%%%%%%%%%%%%%%%%%%%%%%%%%%%%%%%%%%%%%%%%%%%%%%%%%
%%%%%%%%%%%%%%%%%%%%%%%%%%%%%%% PERFORMANCE METRICS %%%%%%%%%%%%%%%%%%%%%%%%%%%%%%%
%%%%%%%%%%%%%%%%%%%%%%%%%%%%%%%%%%%%%%%%%%%%%%%%%%%%%%%%%%%%%%%%%%%%%%%%%%%%%%%%%%%

\subsection{Performance Metrics}

Insert performance metrics here

%%%%%%%%%%%%%%%%%%%%%%%%%%%%%%%%%%%%%%%%%%%%%%%%%%%%%%%%%%%%%%%%%%%%%%%%%%%%%%%%%%%
%%%%%%%%%%%%%%%%%%%%%%%%%%%%%%%%%%% SIMULATION %%%%%%%%%%%%%%%%%%%%%%%%%%%%%%%%%%%%
%%%%%%%%%%%%%%%%%%%%%%%%%%%%%%%%%%%%%%%%%%%%%%%%%%%%%%%%%%%%%%%%%%%%%%%%%%%%%%%%%%%

Insert simulation output here

%%%%%%%%%%%%%%%%%%%%%%%%%%%%%%%%%%%%%%%%%%%%%%%%%%%%%%%%%%%%%%%%%%%%%%%%%%%%%%%%%%%
%%%%%%%%%%%%%%%%%%%%%%%%%%%%%%%%%%% COMPARISONS %%%%%%%%%%%%%%%%%%%%%%%%%%%%%%%%%%%
%%%%%%%%%%%%%%%%%%%%%%%%%%%%%%%%%%%%%%%%%%%%%%%%%%%%%%%%%%%%%%%%%%%%%%%%%%%%%%%%%%%

\subsection{Performance Comparisons}

Insert experiments comparisons here

%%%%%%%%%%%%%%%%%%%%%%%%%%%%%%%%%%%%%%%%%%%%%%%%%%%%%%%%%%%%%%%%%%%%%%%%%%%%%%%%%%%
%%%%%%%%%%%%%%%%%%%%%%%%%%%%%%%%%%% OBSERVATION %%%%%%%%%%%%%%%%%%%%%%%%%%%%%%%%%%%
%%%%%%%%%%%%%%%%%%%%%%%%%%%%%%%%%%%%%%%%%%%%%%%%%%%%%%%%%%%%%%%%%%%%%%%%%%%%%%%%%%%

\subsection{Observation}

Insert observation here


%%%%%%%%%%%%%%%%%%%%%%%%%%%%%%%%%%%%%%%%%%%%%%%%%%%%%%%%%%%%%%%%%%%%%%%%%%%%%%%%%%%
%%%%%%%%%%%%%%%%%%%%%%%%%%%%%%%%%%% CONCLUSION %%%%%%%%%%%%%%%%%%%%%%%%%%%%%%%%%%%%
%%%%%%%%%%%%%%%%%%%%%%%%%%%%%%%%%%%%%%%%%%%%%%%%%%%%%%%%%%%%%%%%%%%%%%%%%%%%%%%%%%%

\section{Conclusion and Future Work}

Insert conclusion and future work here

%%%%%%%%%%%%%%%%%%%%%%%%%%%%%%%%%%%%%%%%%%%%%%%%%%%%%%%%%%%%%%%%%%%%%%%%%%%%%%%%%%%
%%%%%%%%%%%%%%%%%%%%%%%%%%%%%%%%%%%% REFERENCES %%%%%%%%%%%%%%%%%%%%%%%%%%%%%%%%%%%
%%%%%%%%%%%%%%%%%%%%%%%%%%%%%%%%%%%%%%%%%%%%%%%%%%%%%%%%%%%%%%%%%%%%%%%%%%%%%%%%%%%

\begin{thebibliography}{00}

\bibitem{b1} H.S. Behera, Reena Kumari Naik, Suchilagna Parida, ``Improved multilevel feedback queue scheduling using dynamic time quantum and its performance analysis'', \textit{International Journal of Computer Science and Information Technologies}, vol. 3, no. 2, 2012, pp. 3801-3807.

\bibitem{b2} MohammadReza EffatParvar, Karim Faez, Mehdi EffatParvar, Mehdi Zarei, Saeed Safari, ``An intelligent MLFQ scheduling algorithm (IMLFQ) with fault tolerant mechanism'', \textit{Sixth International Conference on Intelligent Systems Design and Applications}, vol. 3, 2006, pp. 80–85.

\bibitem{b3} Rakesh Mohanty, H. S. Behera, Khusbu Patwari, Monisha Dash, ``Design and performance evaluation of a new proposed shortest remaining burst round robin (SRBRR) scheduling algorithm'', \textit{International Symposium on Computer Engineering \& Technology}, vol. 17, 2010.

\bibitem{b4} S. K. Dwivedi and R. Gupta, ``A simulator based performance analysis of multilevel feedback queue scheduling'', \textit{2014 International Conference on Computer and Communication Technology (ICCCT)}, Allahabad, 2014, pp. 341-346.

\bibitem{b5} Malhar Thombare, Rajiv Sukhwani, Priyam Shah, Sheetal Chaudhari, Pooja Raundale, ``Efficient implementation of multilevel feedback queue scheduling'', \textit{2016 International Conference on Wireless Communications, Signal Processing and Networking (WiSPNET)}, Chennai, 2016, pp. 1950–1954.

\bibitem{b6} S. Raheja, R. Dadhich, and S. Rajpalc, ``Designing of vague logic based multilevel feedback queue scheduler'', \textit{Egyptian Informatics Journal}, vol. 17, 2016, pp. 125-137.

\bibitem{b7} A. Alsheikhy, R. Ammar and R. Elfouly, ``An improved dynamic Round Robin scheduling algorithm based on a variant quantum time'', \textit{2015 11th International Computer Engineering Conference (ICENCO)}, Cairo, 2015, pp. 98-104.

\bibitem{b8} M. K. Mishra and F. Rashid, ``An improved round robin CPU scheduling algorithm with varying time quantum'', \textit{International Journal of Computer Science, Engineering and Applications  (IJCSEA)}, vol. 4, 2014.

\bibitem{b9} A. Singh, P. Goyal and S. Batra, ``An optimized round robin scheduling algorithm for CPU scheduling'', \textit{International Journal on Computer Science and Engineering}, 2010, pp. 2383-2385.

\bibitem{b10} Dipto Biswas, Md. Samsuddoha, ``Determining proficient time quantum to  improve the performance  of  round  robin  scheduling  algorithm'', \textit{International  Journal  of  Modern  Education  and  Computer Science(IJMECS)}, vol. 11, 2019, pp. 33-40.

\bibitem{b11} K. Hoganson and J. Brown, ``Real-time scheduling with MLFQ-RT multilevel feedback queue with starvation mitigation'', \textit{2017 International Conference on Engineering, Technology and Innovation (ICE/ITMC)}, Funchal, 2017, pp. 155-160.

\end{thebibliography}

\end{document}
