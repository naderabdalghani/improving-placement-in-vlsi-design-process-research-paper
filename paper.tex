\documentclass[conference]{IEEEtran}
\IEEEoverridecommandlockouts
\usepackage{tikz}
\usetikzlibrary{shapes.geometric, arrows}
\usepackage{cite}
\usepackage{amsmath,amssymb,amsfonts}
\usepackage{algorithm}
\usepackage{algpseudocode}
\usepackage{graphicx}
\usepackage[OT1]{fontenc}
\usepackage{xcolor}
\usepackage{pgfplots}
\pgfplotsset{compat=1.9}

\def\BibTeX{{\rm B\kern-.05em{\sc i\kern-.025em b}\kern-.08em
    T\kern-.1667em\lower.7ex\hbox{E}\kern-.125emX}}

\makeatletter

\begin{document}

%%%%%%%%%%%%%%%%%%%%%%%%%%%%%%%%%%%%%%%%%%%%%%%%%%%%%%%%%%%%%%%%%%%%%%%%%%%%%%%%%%%
%%%%%%%%%%%%%%%%%%%%%%%%%%%%%%%%%%%%%% TITLE %%%%%%%%%%%%%%%%%%%%%%%%%%%%%%%%%%%%%%
%%%%%%%%%%%%%%%%%%%%%%%%%%%%%%%%%%%%%%%%%%%%%%%%%%%%%%%%%%%%%%%%%%%%%%%%%%%%%%%%%%%

\title{Paper Title\\}

%%%%%%%%%%%%%%%%%%%%%%%%%%%%%%%%%%%%%%%%%%%%%%%%%%%%%%%%%%%%%%%%%%%%%%%%%%%%%%%%%%%
%%%%%%%%%%%%%%%%%%%%%%%%%%%%%%%%%%%%% AUTHORS %%%%%%%%%%%%%%%%%%%%%%%%%%%%%%%%%%%%%
%%%%%%%%%%%%%%%%%%%%%%%%%%%%%%%%%%%%%%%%%%%%%%%%%%%%%%%%%%%%%%%%%%%%%%%%%%%%%%%%%%%

\author{\IEEEauthorblockN{Nada Adel\IEEEauthorrefmark{1}, Nader AbdAlGhani\IEEEauthorrefmark{2}, Nourhan Gamal\IEEEauthorrefmark{3} and Sarah Raafat\IEEEauthorrefmark{4}}
\IEEEauthorblockA{Computer Engineering Dept.,
Faculty of Engineering, Cairo University\\
Cairo, Egypt\\
\IEEEauthorrefmark{1}nadoza.98@gmail.com,
\IEEEauthorrefmark{2}nader\_abdelghani@hotmail.com,
\IEEEauthorrefmark{3}nourgamal1498@gmail.com,
\IEEEauthorrefmark{4}sararaafat1@yahoo.com}}


\maketitle

%%%%%%%%%%%%%%%%%%%%%%%%%%%%%%%%%%%%%%%%%%%%%%%%%%%%%%%%%%%%%%%%%%%%%%%%%%%%%%%%%%%
%%%%%%%%%%%%%%%%%%%%%%%%%%%%%%%%%%%% ABSTRACT %%%%%%%%%%%%%%%%%%%%%%%%%%%%%%%%%%%%%
%%%%%%%%%%%%%%%%%%%%%%%%%%%%%%%%%%%%%%%%%%%%%%%%%%%%%%%%%%%%%%%%%%%%%%%%%%%%%%%%%%%

\begin{abstract}

Insert abstract text here

\end{abstract}

%%%%%%%%%%%%%%%%%%%%%%%%%%%%%%%%%%%%%%%%%%%%%%%%%%%%%%%%%%%%%%%%%%%%%%%%%%%%%%%%%%%
%%%%%%%%%%%%%%%%%%%%%%%%%%%%%%%%%%%% KEY WORDS %%%%%%%%%%%%%%%%%%%%%%%%%%%%%%%%%%%%
%%%%%%%%%%%%%%%%%%%%%%%%%%%%%%%%%%%%%%%%%%%%%%%%%%%%%%%%%%%%%%%%%%%%%%%%%%%%%%%%%%%

\begin{IEEEkeywords}

Insert keywords here

\end{IEEEkeywords}

%%%%%%%%%%%%%%%%%%%%%%%%%%%%%%%%%%%%%%%%%%%%%%%%%%%%%%%%%%%%%%%%%%%%%%%%%%%%%%%%%%%
%%%%%%%%%%%%%%%%%%%%%%%%%%%%%%%%%% INTRODUCTION %%%%%%%%%%%%%%%%%%%%%%%%%%%%%%%%%%%
%%%%%%%%%%%%%%%%%%%%%%%%%%%%%%%%%%%%%%%%%%%%%%%%%%%%%%%%%%%%%%%%%%%%%%%%%%%%%%%%%%%

\section{Introduction}

The result of the astronomical impact of the very large scale integration (VLSI) industry on the global economy and the extreme widespread use of electronic devices and equipment in various fields across the globe have led to the exponential increase in demand and sophistication of electronic devices. This, of course, goes along with Moore's law which refers to Gordon Moore's projection in 1965 that the number of transistors crammed in integrated circuits (ICs) doubles every two years, though their cost is halved. This increase in the number of transistors put in an IC comes at a cost which are the problems that arise during both the design process and the fabrication process of such intricate devices. This paper addresses the role of placement in the physical design process. Placement tools are used mainly for two things, the first is to place cells in a suitable way for the consequent routing process, and the second is to guide both synthesis and floorplanning processes. Over the years, placement has no longer become a fine-tunning tool used in physical design, but for that matter, it has become a notable contributor to the timing closure process results as well as being an overall major step in physical design. Conventionally, placement is divided into two subprocesses, global placement and detailed placement, however, the scope of this paper will focus exclusively on the former. Global placement aims to evenly and legally distribute the cells over the available placement region putting in mind cells relative positions to each other globally regardless of local problems that may arise, hence the name. This goal is achieved in favour of certain cost metrics such as overall wirelength and the time taken to achieve such a goal. The placement problem is to place the cells in such a way to achieve the best possible cost metrics. Much like the well-known travelling salesman problem and most, if not all, physical design problems, it is an NP-complete problem in combinatorial optimization. Most of the attempts at solving this problem use heuristic algorithms instead of the ineffective brute force trivial approach. Among those metaheuristics are simulated annealing which is popularly used in the optimization of this particular problem, and genetic algorithm which is used for optimization problems in general. In this paper, we implemented the best of both worlds to achieve overall better optimization in terms of wirelength and time.

%%%%%%%%%%%%%%%%%%%%%%%%%%%%%%%%%%%%%%%%%%%%%%%%%%%%%%%%%%%%%%%%%%%%%%%%%%%%%%%%%%%
%%%%%%%%%%%%%%%%%%%%%%%%%%%%%%%%%% RELATED WORK %%%%%%%%%%%%%%%%%%%%%%%%%%%%%%%%%%%
%%%%%%%%%%%%%%%%%%%%%%%%%%%%%%%%%%%%%%%%%%%%%%%%%%%%%%%%%%%%%%%%%%%%%%%%%%%%%%%%%%%

\section{Related Work}

As already stated, the placement problem is an NP-complete problem where its optimum solution can be found by trying all possible cells configurations, but due to the tremendous number of cells that exist in almost all VLSI designs which range from millions to billions of cells, it would be utterly impractical to compute every possible configuration in order to find the fittest one. Because of this nature, previous literature made significant improvements in terms of solution fitness and the amount of time consumed through using heuristic techniques such as simulated annealing used in \cite{b7}, \cite{b10}, \cite{b11}, \cite{b12}, \cite{b13} and \cite{b14}, min-cut placement algorithm used in \cite{b1} and \cite{b15} and genetic algorithms used in \cite{b2}, \cite{b3} and \cite{b7}. However, optimizing placement algorithms isn't limited to heuristic approaches only as others have used different approaches to achieve solid results. For instance, Natarajan et al \cite{b8} used an analytical placement algorithm that is based on the quadratic placement approach. 

%%%%%%%%%%%%%%%%%%%%%%%%%%%%%%%%%%%%%%%%%%%%%%%%%%%%%%%%%%%%%%%%%%%%%%%%%%%%%%%%%%%
%%%%%%%%%%%%%%%%%%%%%%%%%%%%%%%% PROPOSED APPROACH %%%%%%%%%%%%%%%%%%%%%%%%%%%%%%%%
%%%%%%%%%%%%%%%%%%%%%%%%%%%%%%%%%%%%%%%%%%%%%%%%%%%%%%%%%%%%%%%%%%%%%%%%%%%%%%%%%%%

\section{Proposed Approach}

Insert proposed approach here

%%%%%%%%%%%%%%%%%%%%%%%%%%%%%%%%%%%%%%%%%%%%%%%%%%%%%%%%%%%%%%%%%%%%%%%%%%%%%%%%%%%
%%%%%%%%%%%%%%%%%%%%%%%%%%%%%% EXPERIMENTAL ANALYSIS %%%%%%%%%%%%%%%%%%%%%%%%%%%%%%
%%%%%%%%%%%%%%%%%%%%%%%%%%%%%%%%%%%%%%%%%%%%%%%%%%%%%%%%%%%%%%%%%%%%%%%%%%%%%%%%%%%

\section{Experimental Analysis}

%%%%%%%%%%%%%%%%%%%%%%%%%%%%%%%%%%%%%%%%%%%%%%%%%%%%%%%%%%%%%%%%%%%%%%%%%%%%%%%%%%%
%%%%%%%%%%%%%%%%%%%%%%%%%%%%%%%%%%% ASSUMPTIONS %%%%%%%%%%%%%%%%%%%%%%%%%%%%%%%%%%%
%%%%%%%%%%%%%%%%%%%%%%%%%%%%%%%%%%%%%%%%%%%%%%%%%%%%%%%%%%%%%%%%%%%%%%%%%%%%%%%%%%%

\subsection{Assumptions}

Insert assumptions here

%%%%%%%%%%%%%%%%%%%%%%%%%%%%%%%%%%%%%%%%%%%%%%%%%%%%%%%%%%%%%%%%%%%%%%%%%%%%%%%%%%%
%%%%%%%%%%%%%%%%%%%%%%%%%%%%%%%%%%%%%%% I/O %%%%%%%%%%%%%%%%%%%%%%%%%%%%%%%%%%%%%%%
%%%%%%%%%%%%%%%%%%%%%%%%%%%%%%%%%%%%%%%%%%%%%%%%%%%%%%%%%%%%%%%%%%%%%%%%%%%%%%%%%%%

\subsection{Experimental Scheme}

Insert I/O here

%%%%%%%%%%%%%%%%%%%%%%%%%%%%%%%%%%%%%%%%%%%%%%%%%%%%%%%%%%%%%%%%%%%%%%%%%%%%%%%%%%%
%%%%%%%%%%%%%%%%%%%%%%%%%%%%%%% PERFORMANCE METRICS %%%%%%%%%%%%%%%%%%%%%%%%%%%%%%%
%%%%%%%%%%%%%%%%%%%%%%%%%%%%%%%%%%%%%%%%%%%%%%%%%%%%%%%%%%%%%%%%%%%%%%%%%%%%%%%%%%%

\subsection{Performance Metrics}

Insert performance metrics here

%%%%%%%%%%%%%%%%%%%%%%%%%%%%%%%%%%%%%%%%%%%%%%%%%%%%%%%%%%%%%%%%%%%%%%%%%%%%%%%%%%%
%%%%%%%%%%%%%%%%%%%%%%%%%%%%%%%%%%% SIMULATION %%%%%%%%%%%%%%%%%%%%%%%%%%%%%%%%%%%%
%%%%%%%%%%%%%%%%%%%%%%%%%%%%%%%%%%%%%%%%%%%%%%%%%%%%%%%%%%%%%%%%%%%%%%%%%%%%%%%%%%%

Insert simulation output here

%%%%%%%%%%%%%%%%%%%%%%%%%%%%%%%%%%%%%%%%%%%%%%%%%%%%%%%%%%%%%%%%%%%%%%%%%%%%%%%%%%%
%%%%%%%%%%%%%%%%%%%%%%%%%%%%%%%%%%% COMPARISONS %%%%%%%%%%%%%%%%%%%%%%%%%%%%%%%%%%%
%%%%%%%%%%%%%%%%%%%%%%%%%%%%%%%%%%%%%%%%%%%%%%%%%%%%%%%%%%%%%%%%%%%%%%%%%%%%%%%%%%%

\subsection{Performance Comparisons}

Insert experiments comparisons here

%%%%%%%%%%%%%%%%%%%%%%%%%%%%%%%%%%%%%%%%%%%%%%%%%%%%%%%%%%%%%%%%%%%%%%%%%%%%%%%%%%%
%%%%%%%%%%%%%%%%%%%%%%%%%%%%%%%%%%% OBSERVATION %%%%%%%%%%%%%%%%%%%%%%%%%%%%%%%%%%%
%%%%%%%%%%%%%%%%%%%%%%%%%%%%%%%%%%%%%%%%%%%%%%%%%%%%%%%%%%%%%%%%%%%%%%%%%%%%%%%%%%%

\subsection{Observation}

Insert observation here


%%%%%%%%%%%%%%%%%%%%%%%%%%%%%%%%%%%%%%%%%%%%%%%%%%%%%%%%%%%%%%%%%%%%%%%%%%%%%%%%%%%
%%%%%%%%%%%%%%%%%%%%%%%%%%%%%%%%%%% CONCLUSION %%%%%%%%%%%%%%%%%%%%%%%%%%%%%%%%%%%%
%%%%%%%%%%%%%%%%%%%%%%%%%%%%%%%%%%%%%%%%%%%%%%%%%%%%%%%%%%%%%%%%%%%%%%%%%%%%%%%%%%%

\section{Conclusion and Future Work}

Insert conclusion and future work here

%%%%%%%%%%%%%%%%%%%%%%%%%%%%%%%%%%%%%%%%%%%%%%%%%%%%%%%%%%%%%%%%%%%%%%%%%%%%%%%%%%%
%%%%%%%%%%%%%%%%%%%%%%%%%%%%%%%%%%%% REFERENCES %%%%%%%%%%%%%%%%%%%%%%%%%%%%%%%%%%%
%%%%%%%%%%%%%%%%%%%%%%%%%%%%%%%%%%%%%%%%%%%%%%%%%%%%%%%%%%%%%%%%%%%%%%%%%%%%%%%%%%%

\begin{thebibliography}{00}

\bibitem{b1} Saab, Youssef, ``A fast clustering-based min-cut placement algorithm with simulated-annealing performance'', \textit{VLSI Design}, vol. 5, no. 1, 1996, pp. 37-48.

\bibitem{b2} H. Esbensen, ``A genetic algorithm for macro cell placement'', \textit{European Design Automation Conference}, Hamburg, Germany, 1992, pp. 52-57.

\bibitem{b3} K. Shahookar and P. Mazumder, ``GASP-a genetic algorithm for standard cell placement'', \textit{Proceedings of the European Design Automation Conference}, Glasgow, UK, 1990, pp. 660-664.

\bibitem{b4} J. Vygen, ``Algorithms for detailed placement of standard cells'', \textit{Proceedings Design, Automation and Test in Europe}, Paris, France, 1998, pp. 321-324.

\bibitem{b5} Min Pan, N. Viswanathan and C. Chu, ``An efficient and effective detailed placement algorithm'', \textit{ICCAD-2005. IEEE/ACM International Conference on Computer-Aided Design, 2005}, San Jose, CA, USA, 2005, pp. 48-55.

\bibitem{b6} Z. Jiang, H. Chen, T. Chen and Y. Chang, ``Challenges and Solutions in Modern VLSI Placement'', \textit{2007 International Symposium on VLSI Design, Automation and Test (VLSI-DAT)}, Hsinchu, 2007, pp. 1-5.

\bibitem{b7} Elhaddad, Younis, ``Combined simulated annealing and genetic algorithm to solve optimization problems'', \textit{World Academy of Science, Engineering and Technology}, 2012.

\bibitem{b8} N. Viswanathan and C. C. -. Chu, ``FastPlace: efficient analytical placement using cell shifting, iterative local refinement,and a hybrid net model'', \textit{IEEE Transactions on Computer-Aided Design of Integrated Circuits and Systems}, vol. 24, no. 5, May 2005, pp. 722-733.

\bibitem{b9} D. Zaporozhets, D. Zaruba and N. Kulieva, ``Hybrid heuristic algorithm for VLSI placement'', \textit{2019 International Russian Automation Conference (RusAutoCon)}, Sochi, Russia, 2019, pp. 1-5.

\bibitem{b10} R. F. Hentschke and R. A. D. L. Reis, ``Improving simulated annealing placement by applying random and greedy mixed perturbations [IC layout]'', \textit{16th Symposium on Integrated Circuits and Systems Design, 2003. SBCCI 2003. Proceedings}, Sao Paulo, Brazil, 2003, pp. 267-272.

\bibitem{b11} K. Vorwerk, A. Kennings and J. W. Greene, ``Improving Simulated Annealing-Based FPGA Placement With Directed Moves'', \textit{IEEE Transactions on Computer-Aided Design of Integrated Circuits and Systems}, vol. 28, no. 2, Feb. 2009, pp. 179-192.

\bibitem{b12} Atanu Roy, Karthik Ganesan Pillai, ``Parallel simulated annealing for VLSI
cell placement problem'', 2009.

\bibitem{b13} J. A. Chandy and P. Banerjee, ``Parallel simulated annealing strategies for VLSI cell placement'', \textit{Proceedings of 9th International Conference on VLSI Design}, Bangalore, India, 1996, pp. 37-42.

\bibitem{b14} J. W. Greene and K. J. Supowit, ``Simulated annealing without rejected moves'', \textit{IEEE Transactions on Computer-Aided Design of Integrated Circuits and Systems}, vol. 5, no. 1, January 1986, pp. 221-228.

\bibitem{b15} A. B. Kahng and S. Reda, ``Wirelength minimization for min-cut placements via placement feedback'', \textit{IEEE Transactions on Computer-Aided Design of Integrated Circuits and Systems}, vol. 25, no. 7, July 2006, pp. 1301-1312.

\bibitem{b16} Shahookar, K. and Mazumder, ``VLSI cell placement techniques'', \textit{ACM Computing Surveys}, vol. 23, no. 2, pp. 145-220.

\end{thebibliography}

\end{document}
